\section{行列とその演算} \setcounter{ex}{0}

この章では、線形代数におけるもう一つの主役である\emph{行列}の定義と、基本的な演算について学びます。行列は、複数のベクトルをまとめて扱ったり、ベクトルの変換を表現したりするのに使われます。

\subsection{行列の定義}

\emph{行列}とは、数や文字を長方形に並べたものです。

\begin{dfn}[行列] \label{matrix}
$m$ 行 $n$ 列の行列 $\bm{A}$ は、実数の要素 $a_{ij}$ を $m$ 個の行と $n$ 個の列に並べたもので、次のように表されます。
\[
\bm{A} = \begin{pmatrix}
a_{11} & a_{12} & \cdots & a_{1n} \\
a_{21} & a_{22} & \cdots & a_{2n} \\
\vdots & \vdots & \ddots & \vdots \\
a_{m1} & a_{m2} & \cdots & a_{mn}
\end{pmatrix}
\]
ここで、$a_{ij}$ は行列 $\bm{A}$ の $i$ 行 $j$ 列目の要素(成分)を表します。行列 $\bm{A}$ をその要素を用いて、単に $(a_{ij})$ や $(a_{ij})_{m \times n}$ と略記することもあります。$m$ を\emph{行数}、$n$ を\emph{列数}と呼び、行列のサイズを $m \times n$ 行列と表します。
\end{dfn}

\begin{ex}
\begin{itemize}
\item $2 \times 3$ 行列: $\bm{A} = \begin{pmatrix} 1 & 2 & 3 \\ 4 & 5 & 6 \end{pmatrix}$ (2行3列)
\item $3 \times 1$ 行列: $\bm{b} = \begin{pmatrix} 7 \\ 8 \\ 9 \end{pmatrix}$ (これは3次元の列ベクトルと見なせます)
\item $1 \times 2$ 行列: $\bm{c} = \begin{pmatrix} 10 & 11 \end{pmatrix}$ (これは2次元の行ベクトルと見なせます)
\end{itemize}
\end{ex}

\begin{dfn}[行列の等しさ] \label{matrix_equality}
2つの $m \times n$ 行列 $\bm{A} = (a_{ij})$ と $\bm{B} = (b_{ij})$ が\emph{等しい}とは、それらの対応するすべての要素が等しいことをいいます。すなわち、$a_{ij} = b_{ij}$ がすべての $i=1, \ldots, m$ および $j=1, \ldots, n$ について成り立つとき、$\bm{A} = \bm{B}$ と定義します。
\end{dfn}

\begin{ex}
$\begin{pmatrix} x & y \\ z & w \end{pmatrix} = \begin{pmatrix} 1 & 2 \\ 3 & 4 \end{pmatrix}$ ならば、$x=1,\ y=2,\ z=3,\ w=4$ です。
\end{ex}

\begin{dfn}[零行列] \label{zero_matrix}
すべての要素が $0$ である $m \times n$ 行列を\emph{零行列}といい、$\bm{O}$(または $\bm{O}_{m \times n}$)で表します。
\end{dfn}

\begin{ex}
$2 \times 2$ 零行列: $\bm{O} = \begin{pmatrix} 0 & 0 \\ 0 & 0 \end{pmatrix}$
\end{ex}

\subsection{行列の基本的な演算と性質}

行列には、ベクトルの演算と同様に、和、スカラー倍、そして特有の積(掛け算)が定義されます。

\subsubsection{行列の和(足し算)}

\begin{dfn}[行列の和] \label{matrix_sum}
同じサイズの2つの行列 $\bm{A} = (a_{ij})$ と $\bm{B} = (b_{ij})$ の和 $\bm{A} + \bm{B}$ は、対応する要素同士を足し合わせることで定義されます。和の行列の $(i,j)$ 要素は $a_{ij} + b_{ij}$ です。
\[\bm{A} + \bm{B} = (a_{ij} + b_{ij})\]
\end{dfn}

\begin{rmk*}
行列の和は、同じ行数と同じ列数を持つ行列同士でしか定義できません。
\end{rmk*}

\begin{ex}
$\begin{pmatrix} 1 & 2 \\ 3 & 4 \end{pmatrix} + \begin{pmatrix} 5 & 6 \\ 7 & 8 \end{pmatrix} = \begin{pmatrix} 1+5 & 2+6 \\ 3+7 & 4+8 \end{pmatrix} = \begin{pmatrix} 6 & 8 \\ 10 & 12 \end{pmatrix}$
\end{ex}

\begin{thm}[行列の和の性質] \label{matrix_sum_property}
行列の和は、ベクトルの和と同様に以下の性質を持ちます。
\begin{enumerate}
\item \emph{交換法則}: $\bm{A} + \bm{B} = \bm{B} + \bm{A}$
\item \emph{結合法則}: $(\bm{A} + \bm{B}) + \bm{C} = \bm{A} + (\bm{B} + \bm{C})$
\item \emph{零行列の存在}: 任意の行列 $\bm{A}$ に対して、$\bm{A} + \bm{O} = \bm{A}$ となる零行列 $\bm{O}$ が存在する。
\item \emph{逆行列(加法逆元)の存在}: 任意の行列 $\bm{A}$ に対して、$\bm{A} + (-\bm{A}) = \bm{O}$ となる行列 $-\bm{A}$ が存在する。ここで、$-\bm{A}$ は $\bm{A}$ の各要素に $-1$ を掛けた行列 $(-a_{ij})$ です。
\end{enumerate}
\begin{proof*}
ここでは、\emph{交換法則}の証明を示します。その他の性質は、同様の方法で要素計算によって証明できます(一部は練習問題とします)。\par
$\bm{A} = (a_{ij}),\ \bm{B} = (b_{ij})$ とする。定義\ref{matrix_sum}より、$\bm{A} + \bm{B}$ の $(i,j)$ 要素は $a_{ij} + b_{ij}$ である。実数の足し算は交換法則を満たす($x+y = y+x$)ので、$a_{ij} + b_{ij} = b_{ij} + a_{ij}$ がすべての $i,\ j$ について成り立つ。よって、$\bm{A} + \bm{B}$ の $(i,j)$ 要素は $\bm{B} + \bm{A}$ の $(i,j)$ 要素と等しい。定義\ref{matrix_equality}より、$\bm{A} + \bm{B} = \bm{B} + \bm{A}$ が成り立つ。(証明終)
\end{proof*}
\end{thm}

\subsubsection{行列のスカラー倍(定数倍)と行列の差}

\begin{dfn}[行列のスカラー倍] \label{matrix_scalar}
行列 $\bm{A} = (a_{ij})$ と実数 $k$(スカラー)の積 $k\bm{A}$ は、行列の各要素に $k$ を掛けることで定義されます。積の行列の $(i,j)$ 要素は $k a_{ij}$ です。
\[k\bm{A} = (k a_{ij})\]
\end{dfn}

\begin{ex}
$2 \begin{pmatrix} 1 & 2 \\ 3 & 4 \end{pmatrix} = \begin{pmatrix} 2 \times 1 & 2 \times 2 \\ 2 \times 3 & 2 \times 4 \end{pmatrix} = \begin{pmatrix} 2 & 4 \\ 6 & 8 \end{pmatrix}$
\end{ex}

\begin{thm}[行列のスカラー倍の性質] \label{matrix_scalar_property}
行列のスカラー倍は、以下の性質を持ちます。
\begin{enumerate}
\item \emph{結合法則}: $k(l\bm{A}) = (kl)\bm{A}$
\item \emph{分配法則}: $(k+l)\bm{A} = k\bm{A} + l\bm{A}$ および $k(\bm{A}+\bm{B}) = k\bm{A} + k\bm{B}$
\item \emph{単位元}: $1\bm{A} = \bm{A}$
\item \emph{零元}: $0\bm{A} = \bm{O}$, $k\bm{O} = \bm{O}$
\end{enumerate}
\begin{proof*}
ここでは、\emph{分配法則(行列について)} の証明を示します。その他の性質は、同様の方法で要素計算によって証明できます(一部は練習問題とします)。\par
$\bm{A} = (a_{ij}),\ \bm{B} = (b_{ij}),\ k\in\mathbb{R}$ とする。定義\ref{matrix_sum}より、$\bm{A} + \bm{B}$ の $(i,j)$ 要素は $a_{ij} + b_{ij}$ である。定義\ref{matrix_scalar}より、$k(\bm{A} + \bm{B})$ の $(i,j)$ 要素は $k(a_{ij} + b_{ij})$ である。実数の掛け算は分配法則を満たすので、$k(a_{ij} + b_{ij}) = k a_{ij} + k b_{ij}$ が成り立つ。一方、$k\bm{A}$ の $(i,j)$ 要素は $k a_{ij}$、 $k\bm{B}$ の $(i,j)$ 要素は $k b_{ij}$ である。したがって、$k\bm{A} + k\bm{B}$ の $(i,j)$ 要素は $k a_{ij} + k b_{ij}$ である。両者の $(i,j)$ 要素が等しいため、定義\ref{matrix_equality}より $k(\bm{A}+\bm{B}) = k\bm{A} + k\bm{B}$ が成り立つ。(証明終)
\end{proof*}
\end{thm}

行列の差 $\bm{A} - \bm{B}$ は、\emph{行列の和}と\emph{スカラー倍(加法逆元)}を組み合わせて定義されます。
すなわち、$\bm{A} - \bm{B} = \bm{A} + (-\bm{B})$ です。
これを要素で書くと、$\bm{A} - \bm{B} = (a_{ij} - b_{ij})$ です。

\begin{ex}
$\begin{pmatrix} 5 & 6 \\ 7 & 8 \end{pmatrix} - \begin{pmatrix} 1 & 2 \\ 3 & 4 \end{pmatrix} = \begin{pmatrix} 5-1 & 6-2 \\ 7-3 & 8-4 \end{pmatrix} = \begin{pmatrix} 4 & 4 \\ 4 & 4 \end{pmatrix}$
\end{ex}

\subsubsection{行列の積(掛け算)}

行列の積は、これまでの和やスカラー倍とは異なり、少し複雑な定義を持ちます。しかし、その定義には深い意味があり、線形変換の合成などを自然に表現できます。

\begin{dfn}[行列の積] \label{matrix_product}
$m \times l$ 行列 $\bm{A} = (a_{ij})$ と $l \times n$ 行列 $\bm{B} = (b_{jk})$ の積 $\bm{C} = \bm{A}\bm{B}$ は、$m \times n$ 行列として定義されます。積の行列 $\bm{C}$ の $(i,k)$ 要素 $c_{ik}$ は、$\bm{A}$ の $i$ 行目の要素と $\bm{B}$ の $k$ 列目の要素をそれぞれ掛け合わせて足し合わせたものです。
\[c_{ik} = \sum_{j=1}^{l} a_{ij} b_{jk} = a_{i1}b_{1k} + a_{i2}b_{2k} + \dots + a_{il}b_{lk}\]
\end{dfn}

\begin{rmk*}
行列の積 $\bm{A}\bm{B}$ は、$\bm{A}$ の\emph{列数}と $\bm{B}$ の\emph{行数}が一致する場合にのみ定義されます。結果として得られる行列の行数は $\bm{A}$ の行数、列数は $\bm{B}$ の列数になります。
\[(\text{m} \times \underline{\text{l}}) \times (\underline{\text{l}} \times \text{n}) = (\text{m} \times \text{n})\]
\end{rmk*}

\begin{ex} \label{matrix_product_example}
$\bm{A} = \begin{pmatrix} 1 & 2 \\ 3 & 4 \end{pmatrix}$ ($2 \times 2$ 行列) と $\bm{B} = \begin{pmatrix} 5 & 6 \\ 7 & 8 \end{pmatrix}$ ($2 \times 2$ 行列) の積を計算してみましょう。
\begin{gather*}
\bm{C} = \bm{A}\bm{B} = \begin{pmatrix} 1 & 2 \\ 3 & 4 \end{pmatrix} \begin{pmatrix} 5 & 6 \\ 7 & 8 \end{pmatrix}\\
\begin{aligned}
c_{11} &= (1 \times 5) + (2 \times 7) = 5 + 14 = 19\\
c_{12} &= (1 \times 6) + (2 \times 8) = 6 + 16 = 22\\
c_{21} &= (3 \times 5) + (4 \times 7) = 15 + 28 = 43\\
c_{22} &= (3 \times 6) + (4 \times 8) = 18 + 32 = 50
\end{aligned}
\end{gather*}
よって、
\[\bm{A}\bm{B} = \begin{pmatrix} 19 & 22 \\ 43 & 50 \end{pmatrix}\]
\end{ex}

\begin{thm}[行列の積の性質] \label{matrix_product_property}
行列の積は、以下の性質を持ちます。
\begin{enumerate}
\item \emph{結合法則}: $(\bm{A}\bm{B})\bm{C} = \bm{A}(\bm{B}\bm{C})$(積が定義される限り、計算の順序は問わない)
\item \emph{分配法則}: $\bm{A}(\bm{B} + \bm{C}) = \bm{A}\bm{B} + \bm{A}\bm{C}$ および $(\bm{A} + \bm{B})\bm{C} = \bm{A}\bm{C} + \bm{B}\bm{C}$(和と積の間には分配法則が成り立つ)
\item \emph{スカラー倍との結合法則}: $k(\bm{A}\bm{B}) = (k\bm{A})\bm{B} = \bm{A}(k\bm{B})$(スカラー倍は積のどこに適用しても結果は同じ)
\end{enumerate}
\begin{proof*}
ここでは、\emph{分配法則}の一部である $\bm{A}(\bm{B} + \bm{C}) = \bm{A}\bm{B} + \bm{A}\bm{C}$ の証明を示します。その他の性質も同様に、積の定義と実数の演算法則を用いて要素ごとに証明できます(一部は練習問題とします)。\par
$\bm{A}$ を $m \times l$ 行列、$\bm{B}$ と $\bm{C}$ を $l \times n$ 行列とする。$(\bm{B} + \bm{C})$ の $(j,k)$ 要素は $b_{jk} + c_{jk}$ である(定義\ref{matrix_sum})。積 $\bm{A}(\bm{B} + \bm{C})$ の $(i,k)$ 要素を考えると、定義\ref{matrix_product}より、
\[(\bm{A}(\bm{B} + \bm{C}))_{ik} = \sum_{j=1}^{l} a_{ij}(b_{jk} + c_{jk})= \sum_{j=1}^{l} a_{ij}b_{jk} + \sum_{j=1}^{l} a_{ij}c_{jk}\]
右辺の各項は、それぞれ $\bm{A}\bm{B}$ の $(i,k)$ 要素と $\bm{A}\bm{C}$ の $(i,k)$ 要素の定義(定義\ref{matrix_product})に他ならない。したがって、
\[(\bm{A}(\bm{B} + \bm{C}))_{ik} = (\bm{A}\bm{B})_{ik} + (\bm{A}\bm{C})_{ik}\]
これは、行列の和の定義(定義\ref{matrix_sum})により、$(\bm{A}\bm{B} + \bm{A}\bm{C})$ の $(i,k)$ 要素と等しい。すべての要素が等しいため、定義\ref{matrix_equality}より $\bm{A}(\bm{B} + \bm{C}) = \bm{A}\bm{B} + \bm{A}\bm{C}$ が成り立つ。(証明終)
\end{proof*}
\end{thm}

\begin{rmk*}
行列の積は、一般に\emph{交換法則が成り立ちません}。つまり、$\bm{A}\bm{B} \neq \bm{B}\bm{A}$ であることがほとんどです。
\end{rmk*}

\begin{ex}[交換法則が成り立たない例]
例\ref{matrix_product_example}で $\bm{B}\bm{A}$ を計算してみましょう。
\begin{gather*}
\bm{B}\bm{A} = \begin{pmatrix} 5 & 6 \\ 7 & 8 \end{pmatrix} \begin{pmatrix} 1 & 2 \\ 3 & 4 \end{pmatrix}\\
\begin{aligned}
c'_{11} = (5 \times 1) + (6 \times 3) = 5 + 18 = 23\\
c'_{12} = (5 \times 2) + (6 \times 4) = 10 + 24 = 34\\
c'_{21} = (7 \times 1) + (8 \times 3) = 7 + 24 = 31\\
c'_{22} = (7 \times 2) + (8 \times 4) = 14 + 32 = 46
\end{aligned}
\end{gather*}
よって、
\[\bm{B}\bm{A} = \begin{pmatrix} 23 & 34 \\ 31 & 46 \end{pmatrix}\]
先ほどの $\bm{A}\bm{B}$ とは明らかに異なるため、$\bm{A}\bm{B} \neq \bm{B}\bm{A}$ です。
\end{ex}

\subsubsection{行列の基本変形}

行列の\emph{基本変形}は、行列に対して行われる基本的な操作のことで、行基本変形と列基本変形があります。これらの操作は、行列の多くの重要な性質(例:連立一次方程式の解、階数、逆行列、行列式)を保ちながら、行列をより扱いやすい形に変換するために用いられます。

\begin{dfn}[行基本変形]
行列に行われる以下の3種類の操作を\emph{行基本変形}と呼びます。
\begin{enumerate}
    \item \emph{ある行を $c$ 倍する}: ある行のすべての要素を、ゼロではないスカラー $c$ で掛け合わせる。($R_i \to cR_i,\ c \neq 0$)
    \item \emph{2つの行を入れ替える}: 2つの行の位置を交換する。($R_i \leftrightarrow R_j$)
    \item \emph{ある行に別の行の $c$ 倍を加える}: ある行の各要素に、別の行の対応する要素の $c$ 倍を加える。($R_i \to R_i + cR_j$)
\end{enumerate}
\end{dfn}

\begin{ex}
行列 $\bm{A} = \begin{pmatrix} 1 & 2 & 3 \\ 4 & 5 & 6 \\ 7 & 8 & 9 \end{pmatrix}$ に対して、以下の行基本変形を行ってみましょう。
\begin{enumerate}
    \item 1行目を2倍する:
    \[ \begin{pmatrix} 1 & 2 & 3 \\ 4 & 5 & 6 \\ 7 & 8 & 9 \end{pmatrix} \xrightarrow{2R_1} \begin{pmatrix} 2 & 4 & 6 \\ 4 & 5 & 6 \\ 7 & 8 & 9 \end{pmatrix} \]
    \item 1行目と2行目を入れ替える:
    \[ \begin{pmatrix} 1 & 2 & 3 \\ 4 & 5 & 6 \\ 7 & 8 & 9 \end{pmatrix} \xrightarrow{R_1 \leftrightarrow R_2} \begin{pmatrix} 4 & 5 & 6 \\ 1 & 2 & 3 \\ 7 & 8 & 9 \end{pmatrix} \]
    \item 2行目に1行目の-4倍を加える:
    \[ \begin{pmatrix} 1 & 2 & 3 \\ 4 & 5 & 6 \\ 7 & 8 & 9 \end{pmatrix} \xrightarrow{R_2 \to R_2 - 4R_1} \begin{pmatrix} 1 & 2 & 3 \\ 4-4(1) & 5-4(2) & 6-4(3) \\ 7 & 8 & 9 \end{pmatrix} = \begin{pmatrix} 1 & 2 & 3 \\ 0 & -3 & -6 \\ 7 & 8 & 9 \end{pmatrix} \]
\end{enumerate}
\end{ex}

\begin{dfn}[列基本変形]
行列に行われる以下の3種類の操作を\emph{列基本変形}と呼びます。
\begin{enumerate}
    \item \emph{ある列を $c$ 倍する}: ある列のすべての要素を、ゼロではないスカラー $c$ で掛け合わせる。($C_j \to cC_j,\ c \neq 0$)
    \item \emph{2つの列を入れ替える}: 2つの列の位置を交換する。($C_j \leftrightarrow C_k$)
    \item \emph{ある列に別の列の $c$ 倍を加える}: ある列の各要素に、別の列の対応する要素の $c$ 倍を加える。($C_j \to C_j + cC_k$)
\end{enumerate}
\end{dfn}

\begin{ex}
行列 $\bm{A} = \begin{pmatrix} 1 & 2 & 3 \\ 4 & 5 & 6 \end{pmatrix}$ に対して、以下の列基本変形を行ってみましょう。
\begin{enumerate}
    \item 1列目を2倍する:
    \[ \begin{pmatrix} 1 & 2 & 3 \\ 4 & 5 & 6 \end{pmatrix} \xrightarrow{2C_1} \begin{pmatrix} 2 & 2 & 3 \\ 8 & 5 & 6 \end{pmatrix} \]
    \item 1列目と2列目を入れ替える:
    \[ \begin{pmatrix} 1 & 2 & 3 \\ 4 & 5 & 6 \end{pmatrix} \xrightarrow{C_1 \leftrightarrow C_2} \begin{pmatrix} 2 & 1 & 3 \\ 5 & 4 & 6 \end{pmatrix} \]
    \item 2列目に1列目の-2倍を加える:
    \[ \begin{pmatrix} 1 & 2 & 3 \\ 4 & 5 & 6 \end{pmatrix} \xrightarrow{C_2 \to C_2 - 2C_1} \begin{pmatrix} 1 & 2-2(1) & 3 \\ 4 & 5-2(4) & 6 \end{pmatrix} = \begin{pmatrix} 1 & 0 & 3 \\ 4 & -3 & 6 \end{pmatrix} \]
\end{enumerate}
\end{ex}

\begin{rmk*}
基本変形は、行列の\emph{行空間}や\emph{列空間}、\emph{階数}といった重要な特性を変化させません。特に、行基本変形は連立一次方程式の解集合を変化させないため、ガウスの消去法における主要なツールとなります。また、行列式の計算(第8章)や逆行列の導出(後の章)においても不可欠な操作です。
\end{rmk*}

\subsection{特殊な行列}

行列の中には、特に重要な性質を持つものがあります。

\begin{dfn}[正方行列] \label{square_matrix}
行数と列数が等しい行列を\emph{正方行列}と呼びます。$n \times n$ 行列の場合、「$n$次正方行列」といいます。
\end{dfn}

\begin{ex}
$\begin{pmatrix} 1 & 2 \\ 3 & 4 \end{pmatrix}$ は2次正方行列です。
\end{ex}

\begin{dfn}[転置行列]
$m \times n$ 行列 $\bm{A} = (a_{ij})$ の\emph{転置行列} $\bm{A}^T$(または $\bm{A}'$)は、$\bm{A}$ の行と列を入れ替えることによって得られる $n \times m$ 行列と定義されます。すなわち、$\bm{A}^T$ の $(i,j)$ 要素は $\bm{A}$ の $(j,i)$ 要素に等しい。
\[ \bm{A}^T = (a_{ji}) \]
\end{dfn}

\begin{ex}
$\bm{A} = \begin{pmatrix} 1 & 2 & 3 \\ 4 & 5 & 6 \end{pmatrix}$ の転置行列を求めなさい。
\[ \bm{A}^T = \begin{pmatrix} 1 & 4 \\ 2 & 5 \\ 3 & 6 \end{pmatrix} \]
\end{ex}

\begin{dfn}[三角行列] \label{triangular_matrix}
正方行列 $\bm{A}$ のうち、以下のいずれかの条件を満たすものを\emph{三角行列}と呼びます。
\begin{enumerate}
\item \emph{上三角行列}: 対角成分より下のすべての成分が $0$ である行列。すなわち、$i > j$ のとき $a_{ij} = 0$。
\[ \begin{pmatrix} a_{11} & a_{12} & \cdots & a_{1n} \\ 0 & a_{22} & \cdots & a_{2n} \\ \vdots & \ddots & \ddots & \vdots \\ 0 & \cdots & 0 & a_{nn} \end{pmatrix} \]
\item \emph{下三角行列}: 対角成分より上のすべての成分が $0$ である行列。すなわち、$i < j$ のとき $a_{ij} = 0$。
\[ \begin{pmatrix} a_{11} & 0 & \cdots & 0 \\ a_{21} & a_{22} & \cdots & 0 \\ \vdots & \vdots & \ddots & \vdots \\ a_{n1} & a_{n2} & \cdots & a_{nn} \end{pmatrix} \]
\end{enumerate}
\end{dfn}

\begin{ex}
以下の行列が上三角行列と下三角行列の例です。
\begin{itemize}
\item 上三角行列: $\begin{pmatrix} 1 & 2 & 3 \\ 0 & 4 & 5 \\ 0 & 0 & 6 \end{pmatrix}$
\item 下三角行列: $\begin{pmatrix} 7 & 0 & 0 \\ 8 & 9 & 0 \\ 1 & 2 & 3 \end{pmatrix}$
\end{itemize}
\end{ex}

\begin{dfn}[単位行列] \label{identity_matrix}
正方行列であり、対角成分($i=j$ の要素)がすべて $1$ で、その他の成分がすべて $0$ である行列を\emph{単位行列}と呼び、$\bm{I}$(または $\bm{I}_n$)で表します。
\[\bm{I}_n = \begin{pmatrix}
1 & 0 & \cdots & 0 \\
0 & 1 & \cdots & 0 \\
\vdots & \vdots & \ddots & \vdots \\
0 & 0 & \cdots & 1
\end{pmatrix}\]
\end{dfn}

\begin{thm}[単位行列の性質] \label{identity_matrix_property}
任意の $m \times n$ 行列 $\bm{A}$ に対して、以下の関係が成り立ちます。
\begin{enumerate}
\item $\bm{I}_m \bm{A} = \bm{A}$
\item $\bm{A} \bm{I}_n = \bm{A}$
\end{enumerate}
これは、単位行列が数の掛け算における $1$ のような役割を果たすことを意味します。
\begin{proof*}
ここでは、$\bm{I}_m \bm{A} = \bm{A}$ の証明を示します。\par
$\bm{I}_m$ を $m \times m$ 単位行列、$\bm{A}$ を $m \times n$ 行列とする。積 $\bm{I}_m \bm{A}$ の $(i,k)$ 要素を考える。定義\ref{matrix_product}より、
\[(\bm{I}_m \bm{A})_{ik} = \sum_{j=1}^{m} (\bm{I}_m)_{ij} a_{jk}\]
ここで、単位行列の定義(定義\ref{identity_matrix})より、$ (\bm{I}_m)_{ij} $ は $i=j$ のとき $1$、それ以外のとき $0$ である。したがって、和の中の $j$ について、 $j=i$ の項のみが残り、それ以外の項は $0$ になる。
\[\sum_{j=1}^{m} (\bm{I}_m)_{ij} a_{jk} = (\bm{I}_m)_{ii} a_{ik} = 1 \cdot a_{ik} = a_{ik}\]
これは行列 $\bm{A}$ の $(i,k)$ 要素に等しい。すべての要素が等しいため、定義\ref{matrix_equality}より $\bm{I}_m \bm{A} = \bm{A}$ が成り立つ。(証明終)
\end{proof*}
\end{thm}

\subsection{練習問題}

\begin{quiz}
次の行列を計算しなさい。
\begin{enumerate}
\item $\bm{A} = \begin{pmatrix} 1 & 0 \\ 2 & 3 \end{pmatrix},\ \bm{B} = \begin{pmatrix} 4 & 5 \\ 6 & 7 \end{pmatrix}$ のとき、$\bm{A} + \bm{B}$
\item $\bm{C} = \begin{pmatrix} 3 & -1 & 2 \\ 0 & 4 & 1 \end{pmatrix},\ \bm{D} = \begin{pmatrix} 1 & 2 & -3 \\ 5 & -4 & 0 \end{pmatrix}$ のとき、$\bm{C} - \bm{D}$
\end{enumerate}
\end{quiz}

\begin{quiz}
次のスカラー倍を計算しなさい。
\begin{enumerate}
\item $k = 3,\ \bm{A} = \begin{pmatrix} 1 & -2 \\ 0 & 4 \end{pmatrix}$ のとき、$k\bm{A}$
\item $k = -0.5,\ \bm{B} = \begin{pmatrix} 8 & 10 & 4 \\ -2 & 6 & 0 \end{pmatrix}$ のとき、$k\bm{B}$
\end{enumerate}
\end{quiz}

\begin{quiz}
$\bm{X} = \begin{pmatrix} 1 & 2 \\ 3 & 4 \end{pmatrix},\ \bm{Y} = \begin{pmatrix} 5 & 6 \\ 7 & 8 \end{pmatrix}$ とするとき、次の行列を計算しなさい。
\begin{enumerate}
\item $2\bm{X} + \bm{Y}$
\item $\bm{X} - 3\bm{Y}$
\end{enumerate}
\end{quiz}

\begin{quiz}
次の行列の積を計算しなさい。
\begin{enumerate}
\item $\bm{A} = \begin{pmatrix} 1 & 2 \\ 3 & 4 \end{pmatrix},\ \bm{B} = \begin{pmatrix} 0 & 1 \\ 1 & 0 \end{pmatrix}$ のとき、$\bm{A}\bm{B}$ と $\bm{B}\bm{A}$
\item $\bm{C} = \begin{pmatrix} 1 & 2 & 3 \end{pmatrix},\ \bm{D} = \begin{pmatrix} 4 \\ 5 \\ 6 \end{pmatrix}$ のとき、$\bm{C}\bm{D}$ と $\bm{D}\bm{C}$
\item $\bm{E} = \begin{pmatrix} 2 & 0 \\ -1 & 3 \end{pmatrix},\ \bm{F} = \begin{pmatrix} 1 & -2 \\ 0 & 4 \end{pmatrix}$ のとき、$\bm{E}\bm{F}$
\end{enumerate}
\end{quiz}

\begin{quiz}[定理の証明演習]
定理\ref{matrix_sum_property}、定理\ref{matrix_scalar_property}、定理\ref{matrix_product_property}、定理\ref{identity_matrix_property}の残りの性質を、要素計算を用いて各自で証明しなさい。
\end{quiz}