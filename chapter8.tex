\section{行列式} \setcounter{ex}{0}

この章では、線形代数における非常に重要な概念である\emph{行列式}について学びます。行列式は、正方行列から計算されるスカラー値であり、その行列が線形変換によって「どれだけ空間を拡大・縮小するか」や「向きを反転させるか」といった幾何学的な意味を持ちます。また、連立一次方程式の解の存在や一意性、逆行列の存在条件など、多くの重要な応用があります。

\subsection{行列式の定義}

行列式は、正方行列に対して定義される特別な値です。その定義は再帰的に行われます。まず、行列の要素から作られる「小行列式」と「余因子」を定義します。

\begin{dfn}[小行列式] \label{minor_determinant}
$n$次正方行列 $\bm{A}$ の $i$ 行目と $j$ 列目を削除して得られる $(n-1)$次正方行列を $\bm{A}_{ij}$ と書くとき、その行列式を要素 $a_{ij}$ の\emph{小行列式}と呼び、$M_{ij}$ で表します。
\end{dfn}

\begin{ex}
行列 $\bm{A} = \begin{pmatrix} 1 & 2 & 3 \\ 4 & 5 & 6 \\ 7 & 8 & 9 \end{pmatrix}$ の要素 $a_{11}$ の小行列式 $M_{11}$ を求めなさい。\par
$a_{11}$ は1行1列目の要素なので、行列 $\bm{A}$ から1行目と1列目を削除します。
\[ \bm{A}_{11} = \begin{pmatrix} \cancel{1} & \cancel{2} & \cancel{3} \\ \cancel{4} & 5 & 6 \\ \cancel{7} & 8 & 9 \end{pmatrix} \rightarrow \begin{pmatrix} 5 & 6 \\ 8 & 9 \end{pmatrix} \]
したがって、小行列式 $M_{11}$ はこの $2 \times 2$ 行列の行列式です。(ここではまだ行列式の計算方法を厳密に定義していませんが、後ほど詳しく説明します。)
\[ M_{11} = (5)(9) - (6)(8) = 45 - 48 = -3 \]
\end{ex}

\begin{dfn}[余因子] \label{cofactor}
$n$次正方行列 $\bm{A}$ の要素 $a_{ij}$ の\emph{余因子} $C_{ij}$ は、小行列式 $M_{ij}$ に符号 $(-1)^{i+j}$ を掛けたものと定義されます。
\[ C_{ij} = (-1)^{i+j}M_{ij} \]
この符号は、要素 $(i, j)$ の位置がチェス盤のように「+ - + - ...」と交互に符号が変わることを意味します。
\[
\begin{pmatrix}
+ & - & + & \dots \\
- & + & - & \dots \\
+ & - & + & \dots \\
\vdots & \vdots & \vdots & \ddots
\end{pmatrix}
\]
\end{dfn}

\begin{ex}
先ほどの行列 $\bm{A} = \begin{pmatrix} 1 & 2 & 3 \\ 4 & 5 & 6 \\ 7 & 8 & 9 \end{pmatrix}$ の要素 $a_{11}$ の余因子 $C_{11}$ と、要素 $a_{12}$ の余因子 $C_{12}$ を求めなさい。\par
$M_{11} = -3$ であったので、
\[ C_{11} = (-1)^{1+1}M_{11} = (1)(-3) = -3 \]
次に、$a_{12}$ の小行列式 $M_{12}$ を求めます。
\[ \bm{A}_{12} = \begin{pmatrix} \cancel{1} & \cancel{2} & \cancel{3} \\ 4 & \cancel{5} & 6 \\ 7 & \cancel{8} & 9 \end{pmatrix} \rightarrow \begin{pmatrix} 4 & 6 \\ 7 & 9 \end{pmatrix} \]
(ここでもまだ行列式の計算方法を厳密に定義していませんが、同様に後ほど詳しく説明します。)
\[ M_{12} = (4)(9) - (6)(7) = 36 - 42 = -6 \]
したがって、$C_{12}$ は
\[ C_{12} = (-1)^{1+2}M_{12} = (-1)(-6) = 6 \]
\end{ex}

いよいよ行列式の一般的な定義です。これは、特定の行または列に沿って余因子を「展開」することで計算されます。この定義は再帰的であり、最終的に $1 \times 1$ 行列の行列式に帰着します。

\begin{dfn}[行列式 - 余因子展開による定義] \label{determinant}
$n$次正方行列 $\bm{A}$ の\emph{行列式}は、以下の規則によって定義されます。
\begin{enumerate}
    \item \emph{$1 \times 1$ 行列の場合}: $\bm{A} = (a_{11})$ の行列式は $a_{11}$ である。
    \item \emph{$n > 1$ の行列の場合}: 任意の行 $i$ または任意の列 $j$ を選んで、その行(または列)の要素と対応する余因子の積の和として定義されます。
    \begin{itemize}
        \item $i$ 行に沿った展開:
        \[ \sum_{k=1}^{n} a_{ik}C_{ik} = a_{i1}C_{i1} + a_{i2}C_{i2} + \dots + a_{in}C_{in} \]
        \item $j$ 列に沿った展開:
        \[ \sum_{k=1}^{n} a_{kj}C_{kj} = a_{1j}C_{1j} + a_{2j}C_{2j} + \dots + a_{nj}C_{nj} \]
    \end{itemize}
\end{enumerate}
行列式は、記号 $|\bm{A}|$ または $\det(\bm{A})$ で表されます。どの行または列を選んで展開しても、結果は同じになります。
\end{dfn}

\subsection{具体的なサイズの行列式の計算方法の導出}

余因子展開による定義を用いると、以前に学んだ $2 \times 2$ や $3 \times 3$ 行列の行列式の計算方法が、この一般的な定義から自然に導かれることがわかります。

\begin{thm}[$2 \times 2$ 行列の行列式] \label{determinant2}
$2 \times 2$ 行列 $\bm{A} = \begin{pmatrix} a_{11} & a_{12} \\ a_{21} & a_{22} \end{pmatrix}$ の行列式は、余因子展開により次のように導出されます。
\[ \det(\bm{A}) = a_{11}a_{22} - a_{12}a_{21} \]
\begin{proof*}
例えば、1行目に沿って余因子展開を行います。
\[ \det(\bm{A}) = a_{11}C_{11} + a_{12}C_{12} \]
ここで、$C_{11} = (-1)^{1+1}M_{11} = M_{11}$であり,$M_{11}$ は $a_{11}$ を除いた $1 \times 1$ 行列 $(a_{22})$ の行列式なので、$M_{11} = a_{22}$ です。$C_{12} = (-1)^{1+2}M_{12} = -M_{12}$であり,$M_{12}$ は $a_{12}$ を除いた $1 \times 1$ 行列 $(a_{21})$ の行列式なので、$M_{12} = a_{21}$ です。したがって、
\[ \det(\bm{A}) = a_{11}(a_{22}) + a_{12}(-a_{21}) = a_{11}a_{22} - a_{12}a_{21} \]
と導出されます。
\end{proof*}
\end{thm}

\begin{ex}
$\bm{A} = \begin{pmatrix} 3 & 1 \\ 4 & 2 \end{pmatrix}$ の行列式を計算しなさい。\par
上記の公式を適用すると、
\[ \det(\bm{A}) = (3)(2) - (1)(4) = 6 - 4 = 2 \]
\end{ex}

\begin{thm}[$3 \times 3$ 行列の行列式 - サラスの規則] \label{determinant3}
$3 \times 3$ 行列 $\bm{A} = \begin{pmatrix} a_{11} & a_{12} & a_{13} \\ a_{21} & a_{22} & a_{23} \\ a_{31} & a_{32} & a_{33} \end{pmatrix}$ の行列式は、余因子展開の結果として\emph{サラスの規則}で計算できます。
\[ \det(\bm{A}) = a_{11}a_{22}a_{33} + a_{12}a_{23}a_{31} + a_{13}a_{21}a_{32} - a_{13}a_{22}a_{31} - a_{11}a_{23}a_{32} - a_{12}a_{21}a_{33} \]
\begin{proof*} (概要)
1行目に沿って余因子展開を行うことを考えます。
\[ \det(\bm{A}) = a_{11}C_{11} + a_{12}C_{12} + a_{13}C_{13} \]
ここで、$C_{11}, C_{12}, C_{13}$ はそれぞれ $2 \times 2$ 行列の余因子であり、これらの余因子を $2 \times 2$ 行列の行列式の公式で計算し、代入することでサラスの規則の形が得られます。
\end{proof*}
\end{thm}

\begin{ex}
$\bm{A} = \begin{pmatrix} 1 & 2 & 3 \\ 4 & 5 & 6 \\ 7 & 8 & 9 \end{pmatrix}$ の行列式をサラスの規則を用いて計算しなさい。
\begin{align*}
\det(\bm{A}) &= (1)(5)(9) + (2)(6)(7) + (3)(4)(8) - (3)(5)(7) - (1)(6)(8) - (2)(4)(9)\\
&= 45 + 84 + 96 - 105 - 48 - 72\\
&= 225 - 225 = 0
\end{align*}
\end{ex}

\begin{rmk*}
サラスの規則は $3 \times 3$ 行列にのみ適用できる簡便な方法です。$4 \times 4$ 以上の行列式を計算する際は、余因子展開を繰り返し用いる必要があります。
\end{rmk*}

\subsection{行列式の性質}

行列式には、計算を簡略化したり、行列の特性を理解する上で非常に重要な多くの性質があります。これらの性質は、余因子展開の定義から導かれます。

\begin{thm}[行列式の性質] \label{determinant_property}
$n$次正方行列 $\bm{A}$ の行列式は以下の性質を持ちます。
\begin{enumerate}
    \item \emph{転置行列の行列式}: $\det(\bm{A}^T) = \det(\bm{A})$(この性質により、行に関する性質はすべて列にも適用されることが保証されます。)
    \item \emph{行(または列)の交換}: 行列の2つの行(または列)を交換すると、行列式の符号が反転する。
    \item \emph{同じ行(または列)を持つ行列}: 行列が同じ行(または列)を2つ持つ場合、その行列式は $0$ である。
    \item \emph{スカラー倍}: 行列の1つの行(または列)をスカラー $k$ 倍すると、行列式も $k$ 倍になる。
    \item \emph{和の分配法則}: ある行(または列)が2つのベクトルの和である場合、行列式は2つの行列式の和に分解できる(他の行は不変)。
    \item \emph{行(または列)の線形結合}: ある行(または列)に、別の行(または列)の定数倍を加えても、行列式の値は変わらない。これは、行列式の計算において、行基本変形(列基本変形)が非常に有効であることを意味します。
    \item \emph{零行(または零列)を持つ行列}: 行列がすべての成分が $0$ である行(または列)を1つ以上持つ場合、その行列式は $0$ である。
    \item \emph{三角行列の行列式}: 三角行列の行列式は、対角成分の積に等しい。
    \item \emph{積の行列式}: $\det(\bm{A}\bm{B}) = \det(\bm{A})\det(\bm{B})$(この性質は、行列の積の行列式が個々の行列式の積に等しいことを示しており、線形変換の合成における体積(面積)の変化率として理解できます。)
\end{enumerate}
\begin{proof*}
\begin{enumerate}
    \item \emph{転置行列の行列式}: 行列式の定義(余因子展開)と、転置行列の定義から、行と列の役割を入れ替えても結果が同じになることを示すことで証明できます。特に、$1 \times 1$ 行列の定義が基底となります。
    \item \emph{行(または列)の交換}: 余因子展開の定義から帰納的に証明できます。基底ケースとして $2 \times 2$ 行列で確認でき、高次の行列では展開により低次の行列式に帰着されるため、符号の反転が引き継がれます。
    \item \emph{同じ行(または列)を持つ行列}: もし2つの同じ行を持つ行列の行列式が $D$ であると仮定します。その2つの行を交換しても行列は変化しませんが、性質2により行列式の符号は反転するため、$-D$ となります。したがって、$D = -D$ が成り立ち、これを満たすのは $D=0$ の場合のみです。
    \item \emph{スカラー倍}: 余因子展開の定義 $\sum a_{ik}C_{ik}$ を考えると、ある行 $i$ を $k$ 倍することは、その行の全ての要素 $a_{ik}$ が $k a_{ik}$ に置き換わることを意味します。このとき、展開式の各項も $k$ 倍されるため、行列式全体も $k$ 倍になります。
    \item \emph{和の分配法則}: 和の行(または列)に沿って余因子展開を行うことを考えます。展開式の各項が和の形で表されるため、それを分配することで2つの行列式の和に分解できます。
    \item \emph{行(または列)の線形結合}: この性質は、性質3(同じ行を持つ行列式は0)と性質5(和の分配法則)を組み合わせることで証明できます。ある行に別の行の定数倍を加えた行列式は、元の行列式と、同じ行を持つ行列式(その値は0)の和として表せるため、結果として元の行列式と等しくなります。
    \item \emph{零行(または零列)を持つ行列}: すべての成分が $0$ である行(または列)に沿って余因子展開を行うと、展開式の各項の $a_{ik}$(または $a_{kj}$)がすべて $0$ となるため、行列式全体の値も $0$ になります。
    \item \emph{三角行列の行列式}: 例えば、下三角行列の場合、1行目に沿って余因子展開を行うと、$a_{11}$ の項のみが残り、他の項は $0$ となります。これを繰り返すことで、最終的に対角成分の積が得られます。上三角行列の場合も同様です。
    \item \emph{積の行列式}: この証明は、基本行列の性質や行列式の一意性を利用するなど、学部初級レベルの線形代数ではやや発展的な内容となります。ここでは省略します。
\end{enumerate}
\end{proof*}
\end{thm}

\begin{ex}
    \[\det\begin{pmatrix} 1 & 2 \\ 3 & 4 \end{pmatrix} = (1)(4)-(2)(3) = 4-6=-2\]
    1行目と2行目を交換した行列の行列式は、
    \[\det\begin{pmatrix} 3 & 4 \\ 1 & 2 \end{pmatrix} = (3)(2)-(4)(1) = 6-4=2\]
    となり、符号が反転していることがわかります。
\end{ex}

\begin{ex}
    $\det\begin{pmatrix} 1 & 2 \\ 1 & 2 \end{pmatrix} = (1)(2)-(2)(1) = 2-2=0$
\end{ex}

\begin{ex}
	\[\det\begin{pmatrix} 1 & 2 \\ 3 & 4 \end{pmatrix} = -2\]
    1行目を2倍した行列の行列式は、
    \[\det\begin{pmatrix} 2 & 4 \\ 3 & 4 \end{pmatrix} = (2)(4)-(4)(3) = 8-12=-4\]
    となり、元の行列式の2倍になっていることがわかります。
\end{ex}

\begin{ex}
    $\det\begin{pmatrix} a_{11}+b_{11} & a_{12} \\ a_{21}+b_{21} & a_{22} \end{pmatrix} = \det\begin{pmatrix} a_{11} & a_{12} \\ a_{21} & a_{22} \end{pmatrix} + \det\begin{pmatrix} b_{11} & a_{12} \\ b_{21} & a_{22} \end{pmatrix}$
\end{ex}

\begin{ex}
    \[\det\begin{pmatrix} 1 & 2 \\ 3 & 4 \end{pmatrix} = -2\]
    1行目の3倍を2行目に加える操作を行うと、
    \[\det\begin{pmatrix} 1 & 2 \\ 3+3(1) & 4+3(2) \end{pmatrix} = \det\begin{pmatrix} 1 & 2 \\ 6 & 10 \end{pmatrix} = (1)(10)-(2)(6) = 10-12=-2\]
    となり、行列式の値が変化しないことがわかります。
\end{ex}

\begin{ex}
    $\det\begin{pmatrix} 1 & 2 \\ 0 & 0 \end{pmatrix} = (1)(0)-(2)(0)=0$
\end{ex}

\begin{ex}
    $\det\begin{pmatrix} 1 & 2 & 3 \\ 0 & 4 & 5 \\ 0 & 0 & 6 \end{pmatrix} = (1)(4)(6) = 24$
\end{ex}

\subsection{行列式と正則性}

行列式は、行列に逆行列が存在するかどうか、すなわち行列が\emph{正則}であるかどうかの重要な判定条件を与えます。

\begin{dfn}[正則行列と特異行列] \label{regular_matrix}
正方行列 $\bm{A}$ が\emph{正則である(または可逆である)}とは、$\bm{A}\bm{B} = \bm{B}\bm{A} = \bm{I}$ となる行列 $\bm{B}$ が存在するときに言います。この $\bm{B}$ を $\bm{A}$ の\emph{逆行列}と呼び、$\bm{A}^{-1}$ と表記します。正則でない行列を\emph{特異行列}と呼びます。
\end{dfn}

\begin{thm}[行列式と正則性の関係] \label{determinant_and regular_matrix}
正方行列 $\bm{A}$ が正則であることの必要十分条件は、その行列式が $0$ でないことである。すなわち、
\[ \bm{A} \text{ は正則 } \iff \det(\bm{A}) \neq 0 \]
\begin{proof*}
\begin{itemize}
\item ($\Rightarrow$) $\bm{A}$ が正則ならば $\det(\bm{A}) \neq 0$ であることの証明:\\
$\bm{A}$ が正則であると仮定すると、逆行列 $\bm{A}^{-1}$ が存在し、$\bm{A}\bm{A}^{-1} = \bm{I}$ を満たします。行列式の積の性質(定理\ref{determinant_property}-9)より、$\det(\bm{A}\bm{A}^{-1}) = \det(\bm{A})\det(\bm{A}^{-1})$ が成り立ちます。また、単位行列の行列式は $\det(\bm{I}) = 1$ です。したがって、$\det(\bm{A})\det(\bm{A}^{-1}) = 1$ となります。この等式が成り立つためには、$\det(\bm{A})$ も $\det(\bm{A}^{-1})$ も $0$ であってはなりません。よって、$\det(\bm{A}) \neq 0$ が成り立ちます。
\item ($\Leftarrow$) $\det(\bm{A}) \neq 0$ ならば $\bm{A}$ が正則であることの証明:\\
この証明は、より高度な概念(例えば、余因子行列を用いた逆行列の構成や、ガウスの消去法と基本行列の関係)を必要とします。ここでは概要に留めますが、$\det(\bm{A}) \neq 0$ である場合、行列 $\bm{A}$ は行基本変形によって単位行列に変換できることが知られています。これは、$\bm{A}$ がいくつかの基本行列の積で表せることを意味し、その結果として $\bm{A}$ が逆行列を持つことを導きます。
\end{itemize}
\end{proof*}
\end{thm}

この定理は、連立一次方程式の解の存在と一意性にも密接に関連しています。$\det(\bm{A}) \neq 0$ の場合、連立一次方程式 $\bm{A}\bm{x} = \bm{b}$ は一意の解を持ちます。一方、$\det(\bm{A}) = 0$ の場合、解を持たないか、または無限に多くの解を持つことになります。

\begin{ex}
行列 $\bm{A} = \begin{pmatrix} 1 & k \\ 2 & 4 \end{pmatrix}$ が正則行列でないような $k$ の値を求めなさい。\par
行列 $\bm{A}$ が正則でないということは、$\det(\bm{A}) = 0$ であることを意味します。
\[ \det(\bm{A}) = (1)(4) - (k)(2) = 4 - 2k \]
$4 - 2k = 0$ を解くと、$2k = 4$ より $k = 2$ となります。したがって、$k=2$ のとき、行列 $\bm{A}$ は正則行列ではありません。
\end{ex}

\subsection{行列式の幾何学的意味}

行列式は単なる計算上の値にとどまらず、線形変換が空間の体積(2次元では面積)をどのように変化させるかを示す、明確な幾何学的意味を持ちます。

\begin{thm}[行列式の幾何学的意味] \label{geometric_determinant}
線形写像 $f(\bm{x}) = \bm{A}\bm{x}$ (ここで $\bm{A}$ は $n$次正方行列)によって、もとの図形が変換された図形の体積($n=2$ の場合は面積)は、もとの図形の体積に $|\det(\bm{A})|$ を掛けたものに等しい。また、行列式の符号は、変換によって空間の「向き」が保持されるか反転されるかを示す。
\begin{itemize}
    \item $\det(\bm{A}) > 0$ の場合:向きは保存される(例:回転、拡大・縮小)。
    \item $\det(\bm{A}) < 0$ の場合:向きは反転する(例:鏡映)。
    \item $\det(\bm{A}) = 0$ の場合:変換によって体積が $0$ になる(例:射影)。これは、空間がつぶれて次元が減少することを意味します。
\end{itemize}
\begin{proof*}
単位正方形(または単位立方体)が線形写像 $f(\bm{x}) = \bm{A}\bm{x}$ によって変換された図形は、平行四辺形(または平行六面体)になります。この変換後の図形の体積(面積)は、元の図形の体積(面積)に $|\det(\bm{A})|$ を掛けたものに等しいことが知られています。これは、行列式が線形変換による体積(面積)の拡大・縮小率を表すことに起因します。\par
行列式の符号については、以下のようになります。
\begin{itemize}
    \item $\det(\bm{A}) > 0$ の場合:線形変換は空間の「向き」を保存します。例えば、回転や拡大・縮小のような変換です。変換後の基底ベクトルの順序が元の基底ベクトルの順序と同じ向きを保つことを意味します。
    \item $\det(\bm{A}) < 0$ の場合:線形変換は空間の「向き」を反転させます。例えば、鏡映(反転)のような変換です。変換後の基底ベクトルの順序が元の基底ベクトルの順序と逆向きになることを意味します。
    \item $\det(\bm{A}) = 0$ の場合:変換によって空間が「潰れ」、変換後の図形の体積が $0$ になります。これは、線形写像の像空間の次元が元の空間の次元よりも小さくなることを意味し、例えば平面を直線や点に写すような射影変換がこれに該当します。
\end{itemize}
これらの性質の厳密な証明は、多変量積分の変数変換の公式や、行列の列ベクトルが張る図形の体積を計算する理論に基づきますが、ここでは直感的な理解に留めます。
\end{proof*}
\end{thm}

\begin{ex}
線形写像 $f(\bm{x}) = \bm{A}\bm{x}$ において、$\bm{A} = \begin{pmatrix} 2 & 1 \\ 1 & 3 \end{pmatrix}$ とする。この変換により、単位正方形(頂点 $(0,0), (1,0), (0,1), (1,1)$ で囲まれた面積 $1$ の図形)が変換された図形の面積を求めなさい。\par
まず、行列 $\bm{A}$ の行列式を計算します。
\[ \det(\bm{A}) = (2)(3) - (1)(1) = 6 - 1 = 5 \]
単位正方形の面積は $1$ なので、変換後の図形の面積は $1 \times |\det(\bm{A})| = 1 \times |5| = 5$ である。変換された図形は平行四辺形になります。
\end{ex}

\begin{ex}
線形写像 $f(\bm{x}) = \bm{A}\bm{x}$ において、$\bm{A} = \begin{pmatrix} 1 & 0 \\ 0 & -1 \end{pmatrix}$ ($y$軸に関する鏡映変換)とする。この変換により、単位正方形がどのような図形に変換され、その面積と向きがどうなるか答えなさい。\par
まず、行列 $\bm{A}$ の行列式を計算します。
\[ \det(\bm{A}) = (1)(-1) - (0)(0) = -1 \]
変換後の図形の面積は $1 \times |-1| = 1$ である。面積は変わらない。行列式が負の値なので、向きは反転する。実際、この変換は $y$軸に関して図形を鏡映させるため、向きが反転します。
\end{ex}

\subsection{クラメルの公式(連立一次方程式の解法)}

行列式は、正則行列の連立一次方程式の解を求めるための\emph{クラメルの公式}に応用されます。この公式は、特に変数の数が少ない場合に有用です。

\begin{thm}[クラメルの公式] \label{Cramer's_formula}
$n$元連立一次方程式 $\bm{A}\bm{x} = \bm{b}$ が与えられたとき、もし $\det(\bm{A}) \neq 0$ ならば、一意の解 $\bm{x} = \begin{pmatrix} x_1 \\ x_2 \\ \vdots \\ x_n \end{pmatrix}$ が存在し、その各成分は次のように与えられる。
\[ x_i = \frac{\det(\bm{A}_i)}{\det(\bm{A})} \]
ここで $\bm{A}_i$ は、行列 $\bm{A}$ の $i$ 列目を定数ベクトル $\bm{b}$ で置き換えて得られる行列である。
\begin{proof*}
クラメルの公式の証明には、余因子行列を用いる方法が一般的です。
連立一次方程式 $\bm{A}\bm{x} = \bm{b}$ を考えます。ここで、$\bm{A}$ は正方行列、$\bm{x}$ と $\bm{b}$ は列ベクトルです。$\det(\bm{A}) \neq 0$ のとき、$\bm{A}$ は逆行列 $\bm{A}^{-1}$ を持ちます。$\bm{x} = \bm{A}^{-1}\bm{b}$ となります。ここで、逆行列 $\bm{A}^{-1}$ は、余因子行列 $\text{adj}(\bm{A})$ を用いて $\bm{A}^{-1} = \frac{1}{\det(\bm{A})}\text{adj}(\bm{A})$ と表されます。余因子行列 $\text{adj}(\bm{A})$ は、要素が余因子 $C_{ji}$ である行列 $(C_{ji})$ の転置行列として定義されます。つまり、$\text{adj}(\bm{A})_{ij} = C_{ji}$ です。\par
したがって、$x_i$ は、
\begin{align*}
x_i &= (\bm{A}^{-1}\bm{b})_i = \left(\frac{1}{\det(\bm{A})}\text{adj}(\bm{A})\bm{b}\right)_i = \frac{1}{\det(\bm{A})} \sum_{j=1}^{n} (\text{adj}(\bm{A}))_{ij}b_j \\
&= \frac{1}{\det(\bm{A})} \sum_{j=1}^{n} C_{ji}b_j
\end{align*}
ここで、$\sum_{j=1}^{n} C_{ji}b_j$ は、行列 $\bm{A}_i$ (行列 $\bm{A}$ の $i$ 列目を $\bm{b}$ で置き換えた行列)の $i$ 列目以外の列に沿って余因子展開を行ったものに等しくなります。しかし、より正確には、行列 $\bm{A}_i$ を $i$ 列目で余因子展開したものに相当します。具体的には、$\det(\bm{A}_i)$ は、行列 $\bm{A}_i$ の $i$ 列目を基準に余因子展開すると、
\[ \det(\bm{A}_i) = \sum_{j=1}^{n} (\bm{A}_i)_{ji} C'_{ji} \]
ここで $(\bm{A}_i)_{ji}$ は行列 $\bm{A}_i$ の $j$ 行 $i$ 列目の要素であり、これは $b_j$ に等しいです。また、$C'_{ji}$ は行列 $\bm{A}_i$ の $j$ 行 $i$ 列目の余因子ですが、$\bm{A}_i$ は $\bm{A}$ の $i$ 列目だけが異なるため、$C'_{ji}$ は元の行列 $\bm{A}$ の $C_{ji}$ と等しくなります($i$ 列目を取り除いた小行列は同じになるため)。よって、
\[ \det(\bm{A}_i) = \sum_{j=1}^{n} b_j C_{ji} \]
この式と、$x_i$ の式を比較すると、
\[ x_i = \frac{1}{\det(\bm{A})} \sum_{j=1}^{n} b_j C_{ji} = \frac{\det(\bm{A}_i)}{\det(\bm{A})} \]
が成り立ちます。
\end{proof*}
\end{thm}

\begin{ex}
次の連立一次方程式をクラメルの公式を用いて解きなさい。
\begin{align*} 2x + 3y &= 7 \\ x - y &= 1 \end{align*}
まず、この連立方程式を行列の形で書くと $\begin{pmatrix} 2 & 3 \\ 1 & -1 \end{pmatrix} \begin{pmatrix} x \\ y \end{pmatrix} = \begin{pmatrix} 7 \\ 1 \end{pmatrix}$ となります。係数行列を $\bm{A} = \begin{pmatrix} 2 & 3 \\ 1 & -1 \end{pmatrix}$、定数ベクトルを $\bm{b} = \begin{pmatrix} 7 \\ 1 \end{pmatrix}$ とします。
\begin{enumerate}
\item 係数行列 $\bm{A}$ の行列式を計算します。
\[ \det(\bm{A}) = \det\begin{pmatrix} 2 & 3 \\ 1 & -1 \end{pmatrix} = (2)(-1) - (3)(1) = -2 - 3 = -5 \]
$\det(\bm{A}) = -5 \neq 0$ なので、解は一意に存在します。
\item $x$ の係数である1列目を $\bm{b}$ で置き換えた行列 $\bm{A}_1$ の行列式を計算します。
\begin{align*}
\bm{A}_1 &= \begin{pmatrix} 7 & 3 \\ 1 & -1 \end{pmatrix} \\
\det(\bm{A}_1) &= (7)(-1) - (3)(1) = -7 - 3 = -10
\end{align*}
\item $y$ の係数である2列目を $\bm{b}$ で置き換えた行列 $\bm{A}_2$ の行列式を計算します。
\begin{align*}
\bm{A}_2 &= \begin{pmatrix} 2 & 7 \\ 1 & 1 \end{pmatrix} \\
\det(\bm{A}_2) &= (2)(1) - (7)(1) = 2 - 7 = -5
\end{align*}
\item クラメルの公式を適用して $x$ と $y$ の値を求めます。
\begin{align*}
x &= \frac{\det(\bm{A}_1)}{\det(\bm{A})} = \frac{-10}{-5} = 2 \\
y &= \frac{\det(\bm{A}_2)}{\det(\bm{A})} = \frac{-5}{-5} = 1
\end{align*}
よって、連立一次方程式の解は $x=2,\ y=1$ である。
\end{enumerate}
\end{ex}

\subsection{練習問題}

\begin{quiz}
次の行列の行列式を余因子展開を行って計算しなさい。
\begin{enumerate}
    \item $\bm{A} = \begin{pmatrix} 5 & 2 \\ 7 & 3 \end{pmatrix}$
    \item $\bm{B} = \begin{pmatrix} 1 & -1 & 0 \\ 2 & 3 & 4 \\ -1 & 0 & 2 \end{pmatrix}$ 
\end{enumerate}
\end{quiz}

\begin{quiz}
行列 $\bm{A} = \begin{pmatrix} 1 & k \\ 2 & 4 \end{pmatrix}$ が正則行列でないような $k$ の値を求めなさい。
\end{quiz}

\begin{quiz}
線形写像 $f(\bm{x}) = \bm{A}\bm{x}$ において、$\bm{A} = \begin{pmatrix} 3 & 0 \\ 0 & 2 \end{pmatrix}$ とする。この変換により、面積 $1$ の図形がどのような面積の図形に変換されるか答えなさい。
\end{quiz}

\begin{quiz}
クラメルの公式を用いて、次の連立一次方程式を解きなさい。
\begin{align*} 3x - y &= 5 \\ x + 2y &= -1 \end{align*}
\end{quiz}