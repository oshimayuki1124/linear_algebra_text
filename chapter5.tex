\section{基底と次元}
この章では、ベクトル空間の構造を理解するための核心的な概念である\emph{基底}と\emph{次元}について学びます。基底は、ベクトル空間内の全てのベクトルを表現するための「最小限の構成要素」であり、次元はその空間の「大きさ」を示します。
\subsection{基底(Basis)}
これまでに、いくつかのベクトルを線形結合することで、様々なベクトルを表現できることを学びました。基底は、この「表現」を最も効率的に行うための特別なベクトルの集まりです。
\begin{dfn}[基底]
ベクトル空間 $V$ のベクトルの集まり $B = \{\bm{v}_1, \bm{v}_2, \ldots, \bm{v}_n\}$ が\emph{基底である}とは、以下の2つの条件を満たすときに言います。
\begin{enumerate}
\item $B$ が\emph{線形独立である}。
\item $B$ が $V$ を\emph{生成する}(つまり、$\text{span}(B) = V$ である)。
\end{enumerate}
\end{dfn}

\emph{幾何学的解釈}:
\begin{itemize}
\item \emph{線形独立であること}: 基底を構成するベクトルは、それぞれが新しい「方向」を示しており、どれも他のベクトルの組み合わせで表すことができません。これにより、無駄なベクトルが含まれていないことが保証されます。
\item \emph{生成すること}: 基底を構成するベクトルを組み合わせることで、そのベクトル空間内の\emph{すべての}ベクトルを表現できます。これは、空間内のどこへでも「到達できる」ことを意味します。
\end{itemize}

簡単に言えば、基底とは「\emph{最小限の数のベクトルで、その空間のあらゆるベクトルを表せる集まり}」です。

\begin{ex}
\begin{itemize}
\item \emph{2次元空間 $\mathbb{R}^2$ の標準基底}:$\bm{e}_1 = \begin{pmatrix} 1 \\ 0 \end{pmatrix},\ \bm{e}_2 = \begin{pmatrix} 0 \\ 1 \end{pmatrix}$\\
	この集まり $B = \{\bm{e}_1, \bm{e}_2\}$ は $\mathbb{R}^2$ の基底です。
	\begin{itemize}
	\item \emph{線形独立性}: $c_1 \bm{e}_1 + c_2 \bm{e}_2 = \bm{0}$ ならば $c_1=0,\ c_2=0$ の場合に限られるため、線形独立です。
    \item \emph{生成性}: 任意の2次元ベクトル $\bm{x} = \begin{pmatrix} x_1 \\ x_2 \end{pmatrix}$ は $x_1 \bm{e}_1 + x_2 \bm{e}_2$ と書けるため、$\mathbb{R}^2$ を生成します。
    \end{itemize}
\item \emph{別の基底の例($\mathbb{R}^2$)}:$\bm{v}_1 = \begin{pmatrix} 1 \\ 1 \end{pmatrix},\ \bm{v}_2 = \begin{pmatrix} -1 \\ 1 \end{pmatrix}$\\
    この集まり $B' = \{\bm{v}_1, \bm{v}_2\}$ も $\mathbb{R}^2$ の基底です。
	\begin{itemize}
    \item \emph{線形独立性}: $c_1 \bm{v}_1 + c_2 \bm{v}_2 = \bm{0}$ を解くと $c_1=0,\ c_2=0$ しか解がないため、線形独立です。
    \item \emph{生成性}: 任意の $\bm{x} = \begin{pmatrix} x_1 \\ x_2 \end{pmatrix}$ に対して $c_1 \bm{v}_1 + c_2 \bm{v}_2 = \bm{x}$ を満たす $c_1, c_2$ が一意に存在するため、生成します。
    \end{itemize}
\end{itemize}
\end{ex}

\begin{thm}[基底の存在と一意性]
任意の有限次元ベクトル空間には基底が存在します。また、あるベクトル空間の基底を構成するベクトルの個数は、どの基底を選んだとしても常に同じです。
\begin{proof*}[アイデア(個数の一意性)]
この定理の「基底を構成するベクトルの個数は常に同じ」という部分が、次に定義する「次元」の概念を可能にします。その証明のアイデアは、以下に基づきます。
もし、あるベクトル空間に、異なる個数のベクトルを持つ2つの基底 $B_1 = \{\bm{u}_1, \ldots, \bm{u}_m\}$ と $B_2 = \{\bm{v}_1, \ldots, \bm{v}_n\}$ が存在すると仮定します。このとき、$m \neq n$ だとすると、例えば $m < n$ の場合、基底 $B_1$ のベクトルを使って基底 $B_2$ のベクトルを全て線形結合で表そうとすると、ある段階で線形従属な関係が生じてしまい、矛盾が生じることを示します。これにより、$m=n$ でなければならない、と結論づけます。
\end{proof*}
\end{thm}

\subsection{次元(Dimension)}
基底を構成するベクトルの個数が常に同じであるという定理 5.1 の性質は、ベクトル空間の「大きさ」を定量的に表すことを可能にします。
\begin{dfn}[次元]
ベクトル空間 $V$ の\emph{次元 (dimension)} は、その基底を構成するベクトルの個数として定義されます。$V$ の次元は $\dim(V)$ と表記されます。
\end{dfn}

\begin{ex}
\begin{itemize}
\item \emph{2次元空間 $\mathbb{R}^2$}: 標準基底 $\left\{\begin{pmatrix} 1 \\ 0 \end{pmatrix}, \begin{pmatrix} 0 \\ 1 \end{pmatrix}\right\}$ は2つのベクトルからなるので、$\dim(\mathbb{R}^2) = 2$ です。
\item \emph{3次元空間 $\mathbb{R}^3$}: 標準基底 $\left\{\begin{pmatrix} 1 \\ 0 \\ 0 \end{pmatrix}, \begin{pmatrix} 0 \\ 1 \\ 0 \end{pmatrix}, \begin{pmatrix} 0 \\ 0 \\ 1 \end{pmatrix}\right\}$ は3つのベクトルからなるので、$\dim(\mathbb{R}^3) = 3$ です。
\item \emph{$n$次元空間 $\mathbb{R}^n$}: 標準基底は $n$ 個のベクトルからなるので、$\dim(\mathbb{R}^n) = n$ です。
\item \emph{零ベクトル空間 $\{\bm{0}\}$}: この空間の基底は空集合(ベクトルを一つも持たない集合)であると定義されます。そのため、その次元は $0$ です。
\end{itemize}
\end{ex}

\subsection{基底と次元に関する重要な定理}
基底と次元の概念は、ベクトル空間内のベクトルの数や線形独立性・生成性と密接に関わっています。
\begin{thm}
$n$次元ベクトル空間 $V$ において、以下のことが言えます。
\begin{enumerate}
\item $n$個のベクトルからなる線形独立な集まりは、必ず $V$ の基底である。
\item $n$個のベクトルからなる $V$ を生成する集まりは、必ず $V$ の基底である。
\item $V$ 内の $n$個より多いベクトルは、必ず線形従属である。
\item $V$ 内の $n$個より少ないベクトルは、決して $V$ を生成しない。
\end{enumerate}
\begin{proof*}[アイデア(定理 5.2 の3.と4.)]
これらの定理は、基底の持つ性質(線形独立性と生成性、そして個数の一意性)から論理的に導かれます。例えば、3.「$n$個より多いベクトルは必ず線形従属」は、もしそれらが線形独立であると仮定すると、次元が $n$ であることと矛盾することを示すことで証明できます。
\end{proof*}
\end{thm}

\subsection{練習問題}
\begin{quiz}
次のベクトルの集まりが、与えられたベクトル空間の基底であるかどうかを判定しなさい。
\begin{enumerate}
\item 空間 $\mathbb{R}^2$に対して、$\left\{ \begin{pmatrix} 1 \\ 2 \end{pmatrix}, \begin{pmatrix} 3 \\ 4 \end{pmatrix} \right\}$
\item 空間 $\mathbb{R}^3$に対して、$\left\{ \begin{pmatrix} 1 \\ 0 \\ 0 \end{pmatrix}, \begin{pmatrix} 0 \\ 1 \\ 0 \end{pmatrix} \right\}$
\item 空間 $\mathbb{R}^3$に対して、$\left\{ \begin{pmatrix} 1 \\ 0 \\ 0 \end{pmatrix}, \begin{pmatrix} 0 \\ 1 \\ 0 \end{pmatrix}, \begin{pmatrix} 0 \\ 0 \\ 1 \end{pmatrix}, \begin{pmatrix} 1 \\ 1 \\ 1 \end{pmatrix} \right\}$
\end{enumerate}
\end{quiz}

\begin{quiz}
ベクトル空間 $V = \text{span}\left\{ \begin{pmatrix} 1 \\ 1 \\ 0 \end{pmatrix}, \begin{pmatrix} 0 \\ 1 \\ 1 \end{pmatrix}, \begin{pmatrix} 1 \\ 2 \\ 1 \end{pmatrix} \right\}$ の次元を求めなさい。
\end{quiz}