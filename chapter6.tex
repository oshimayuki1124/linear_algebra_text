\section{線形写像と表現行列}

この章では、ベクトル空間間の「変換」である\emph{線形写像}と、その変換を具体的な数値で表す\emph{表現行列}について学びます。これは、行列が単なる数の集まりではなく、空間を動かす「働き」を持つことを理解する上で非常に重要です。

\subsection{線形写像の定義}

線形写像は、ベクトル空間の構造(和とスカラー倍)を保つ特別な種類の関数です。

\begin{dfn}[線形写像]
2つのベクトル空間 $V$ と $W$ が与えられたとき、写像(関数)$f: V \to W$ が\emph{線形写像 (linear map)} であるとは、以下の2つの条件を満たすときに言います。
\begin{enumerate}
\item \emph{加法性}: 任意の $\bm{u}, \bm{v} \in V$ に対して、$f(\bm{u} + \bm{v}) = f(\bm{u}) + f(\bm{v})$
\item \emph{斉次性}: 任意の $\bm{u} \in V$ と任意のスカラー $c$ に対して、$f(c\bm{u}) = c f(\bm{u})$
\end{enumerate}
これらの2つの条件は、まとめて以下の1つの条件で表すこともできます。
\[\text{任意の $\bm{u}, \bm{v} \in V$ と任意のスカラー $c_1, c_2$ に対して}f(c_1\bm{u} + c_2\bm{v}) = c_1 f(\bm{u}) + c_2 f(\bm{v})\]
\end{dfn}

\emph{幾何学的解釈}:
線形写像は、空間を「まっすぐ」かつ「原点を原点に写す」変換だと考えることができます。具体的には、
\begin{itemize}
\item 直線を直線に写します。
\item 原点を通る平面を、原点を通る直線や平面に写します。
\item ベクトルの平行性や等間隔性を保ちます。
\end{itemize}

線形写像の例としては、回転、拡大・縮小、射影(影を落とすような変換)などがあります。

\begin{ex}
\begin{enumerate}
\item \emph{2次元の回転}:\\
    $f: \mathbb{R}^2 \to \mathbb{R}^2$ を、ベクトルを角度 $\theta$ だけ反時計回りに回転させる写像とします。
    $\bm{x} = \begin{pmatrix} x \\ y \end{pmatrix}$ としたとき、$f(\bm{x}) = \begin{pmatrix} x\cos\theta - y\sin\theta \\ x\sin\theta + y\cos\theta \end{pmatrix}$ と表されます。
    この写像が線形写像であることを確認してみましょう。
	\begin{itemize}
	\item \emph{加法性}: $f(\bm{u} + \bm{v})$ と $f(\bm{u}) + f(\bm{v})$ が等しいことを示します。
    \item \emph{斉次性}: $f(c\bm{u})$ と $c f(\bm{u})$ が等しいことを示します。
    \end{itemize}
    これは定義 6.1 の2つの条件を満たすので、回転写像は線形写像です。
\item \emph{スカラー倍(拡大・縮小)}:
    $f: \mathbb{R}^n \to \mathbb{R}^n$ を、$f(\bm{x}) = k\bm{x}$ ($k$は定数スカラー)と定義する写像は線形写像です。
    これは、ベクトルのスカラー倍の性質(定理1.2の2と1.2の3)から明らかです。
\end{enumerate}
\end{ex}

\begin{thm}[線形写像の性質]
線形写像 $f: V \to W$ は、以下の性質を持ちます。
\begin{enumerate}
\item 零ベクトルを零ベクトルに写す: $f(\bm{0}_V) = \bm{0}_W$
    (ここで $\bm{0}_V$ はベクトル空間 $V$ の零ベクトル、$\bm{0}_W$ は $W$ の零ベクトル)
\item 線形結合を線形結合に写す: $f(c_1\bm{v}_1 + \dots + c_k\bm{v}_k) = c_1 f(\bm{v}_1) + \dots + c_k f(\bm{v}_k)$
\end{enumerate}
\begin{proof*}
ここでは、零ベクトルを零ベクトルに写すことの証明を示します。2. の証明は、線形写像の定義を繰り返し適用することで証明できます(練習問題とします)。\par
$V$ の任意のベクトル $\bm{v}$ を考える。
$V$ における零ベクトルの性質(定理 1.1-3)より、$\bm{v} + \bm{0}_V = \bm{v}$ である。
この両辺に $f$ を作用させると、$f(\bm{v} + \bm{0}_V) = f(\bm{v})$ となる。
線形写像の加法性(定義 6.1-1)より、$f(\bm{v} + \bm{0}_V) = f(\bm{v}) + f(\bm{0}_V)$ となる。
したがって、$f(\bm{v}) + f(\bm{0}_V) = f(\bm{v})$ が成り立つ。
$W$ における零ベクトルの性質(定理 1.1-3)により、$f(\bm{0}_V)$ は $W$ の零ベクトルである $\bm{0}_W$ でなければならない。
よって、$f(\bm{0}_V) = \bm{0}_W$ が成り立つ。
(証明終)
\end{proof*}
\end{thm}

\subsection{表現行列}
線形写像の素晴らしい点の1つは、それを\emph{行列}を使って表現できることです。これにより、抽象的な写像の操作を、具体的な行列計算に落とし込むことができます。

\begin{dfn}[表現行列]
$n$次元ベクトル空間 $V$ の基底を $B_V = \{\bm{v}_1, \bm{v}_2, \ldots, \bm{v}_n\}$ とし、$m$次元ベクトル空間 $W$ の基底を $B_W = \{\bm{w}_1, \bm{w}_2, \ldots, \bm{w}_m\}$ とする。
線形写像 $f: V \to W$ の\emph{表現行列 (representation matrix)} $\bm{A}$ とは、各基底ベクトル $f(\bm{v}_j)$ の $W$ の基底 $B_W$ による線形結合の係数を列ベクトルとして並べた $m \times n$ 行列のことです。
\end{dfn}

具体的には、$V$ の各基底ベクトル $\bm{v}_j$ の像 $f(\bm{v}_j)$ を $W$ の基底 $B_W$ で表すと、
\begin{align*}
f(\bm{v}_1) &= a_{11}\bm{w}_1 + a_{21}\bm{w}_2 + \dots + a_{m1}\bm{w}_m\\
f(\bm{v}_2) &= a_{12}\bm{w}_1 + a_{22}\bm{w}_2 + \dots + a_{m2}\bm{w}_m\\
\vdots \\
f(\bm{v}_n) &= a_{1n}\bm{w}_1 + a_{2n}\bm{w}_2 + \dots + a_{mn}\bm{w}_m
\end{align*}

このとき、これらの係数 $a_{ij}$ を並べた行列が表現行列 $\bm{A}$ です。
\[
\bm{A} = \begin{pmatrix}
a_{11} & a_{12} & \cdots & a_{1n} \\
a_{21} & a_{22} & \cdots & a_{2n} \\
\vdots & \vdots & \ddots & \vdots \\
a_{m1} & a_{m2} & \cdots & a_{mn}
\end{pmatrix}
\]
この行列 $\bm{A}$ の $j$ 列目は、$f(\bm{v}_j)$ の $B_W$ に関する座標ベクトルになります。

\begin{thm}[線形写像の行列による表現]
線形写像 $f: V \to W$ の基底 $B_V$ と $B_W$ に関する表現行列を $\bm{A}$ とする。
このとき、任意のベクトル $\bm{x} \in V$ を基底 $B_V$ で表した座標ベクトルを $[\bm{x}]_{B_V}$ とし、$f(\bm{x}) \in W$ を基底 $B_W$ で表した座標ベクトルを $[f(\bm{x})]_{B_W}$ とすると、以下の関係が成り立ちます。
\[[f(\bm{x})]_{B_W} = \bm{A} [\bm{x}]_{B_V}\]
\begin{proof*}[アイデア]
この定理の証明は、線形写像の線形性(定義 6.1)と、基底によるベクトルの表現の一意性(定義 5.1 と定理 5.1)に基づいて行われます。ベクトル $\bm{x}$ を基底 $B_V$ の線形結合で表し、それに $f$ を作用させると、線形性により $f(\bm{x})$ が基底ベクトルの像 $f(\bm{v}_j)$ の線形結合になります。そして、これらの $f(\bm{v}_j)$ を基底 $B_W$ で表し直すことで、結果的に行列の積の形にまとめられることを示します。
\end{proof*}
\end{thm}

\begin{ex}[2次元平面上の回転写像の表現行列]
$\theta = \pi/2$ (90度)の回転写像 $f: \mathbb{R}^2 \to \mathbb{R}^2$ を考えます。
標準基底 $B = \{\bm{e}_1, \bm{e}_2\} = \left\{ \begin{pmatrix} 1 \\ 0 \end{pmatrix}, \begin{pmatrix} 0 \\ 1 \end{pmatrix} \right\}$ を使います。
\begin{enumerate}
\item 基底ベクトル $\bm{e}_1$ の像を計算します:
    $f(\bm{e}_1) = f\begin{pmatrix} 1 \\ 0 \end{pmatrix} = \begin{pmatrix} 1\cos(\pi/2) - 0\sin(\pi/2) \\ 1\sin(\pi/2) + 0\cos(\pi/2) \end{pmatrix} = \begin{pmatrix} 0 \\ 1 \end{pmatrix}$
    これを基底 $B$ で表すと、$0\bm{e}_1 + 1\bm{e}_2$ となるので、座標ベクトルは $\begin{pmatrix} 0 \\ 1 \end{pmatrix}$ です。
\item 基底ベクトル $\bm{e}_2$ の像を計算します:
    $f(\bm{e}_2) = f\begin{pmatrix} 0 \\ 1 \end{pmatrix} = \begin{pmatrix} 0\cos(\pi/2) - 1\sin(\pi/2) \\ 0\sin(\pi/2) + 1\cos(\pi/2) \end{pmatrix} = \begin{pmatrix} -1 \\ 0 \end{pmatrix}$
    これを基底 $B$ で表すと, $-1\bm{e}_1 + 0\bm{e}_2$ となるので、座標ベクトルは $\begin{pmatrix} -1 \\ 0 \end{pmatrix}$ です。
\end{enumerate}

これらの座標ベクトルを列として並べると、回転写像の表現行列 $\bm{A}$ が得られます。
\[\bm{A} = \begin{pmatrix} 0 & -1 \\ 1 & 0 \end{pmatrix}\]
この行列を使えば、任意のベクトル $\bm{x} = \begin{pmatrix} x \\ y \end{pmatrix}$ を回転させた結果を、単に行列の積で計算できます。$$f(\bm{x}) = \begin{pmatrix} 0 & -1 \\ 1 & 0 \end{pmatrix} \begin{pmatrix} x \\ y \end{pmatrix} = \begin{pmatrix} -y \\ x \end{pmatrix}$$
これは確かに、$(x,y)$ を90度回転させた結果と一致します。
\end{ex}

\subsection{練習問題}
\begin{quiz}
次の写像が線形写像であるかどうかを判定しなさい。線形写像である場合は、その証明も簡潔に示しなさい。
\begin{enumerate}
\item $f: \mathbb{R}^2 \to \mathbb{R}^2$, $f\begin{pmatrix} x \\ y \end{pmatrix} = \begin{pmatrix} x+y \\ x-y \end{pmatrix}$
\item $g: \mathbb{R}^2 \to \mathbb{R}$, $g\begin{pmatrix} x \\ y \end{pmatrix} = x^2 + y^2$
\item $h: \mathbb{R}^3 \to \mathbb{R}^2$, $h\begin{pmatrix} x \\ y \\ z \end{pmatrix} = \begin{pmatrix} x+2y \\ z \end{pmatrix}$
\end{enumerate}
\end{quiz}

\begin{quiz}
線形写像 $f: \mathbb{R}^2 \to \mathbb{R}^2$ が、標準基底 $\left\{ \bm{e}_1=\begin{pmatrix} 1 \\ 0 \end{pmatrix}, \bm{e}_2=\begin{pmatrix} 0 \\ 1 \end{pmatrix} \right\}$ に対して、$f(\bm{e}_1) = \begin{pmatrix} 2 \\ 1 \end{pmatrix}$、$f(\bm{e}_2) = \begin{pmatrix} -1 \\ 3 \end{pmatrix}$ を満たすとき、この線形写像の表現行列を求めなさい。また、この行列を使って $f\begin{pmatrix} 4 \\ -2 \end{pmatrix}$ を計算しなさい。
\end{quiz}

\begin{quiz}
線形写像 $f: \mathbb{R}^3 \to \mathbb{R}^2$ が、$f\begin{pmatrix} x \\ y \\ z \end{pmatrix} = \begin{pmatrix} x+y \\ y-z \end{pmatrix}$ で与えられるとき、$\mathbb{R}^3$ の標準基底から $\mathbb{R}^2$ の標準基底への表現行列を求めなさい。
\end{quiz}